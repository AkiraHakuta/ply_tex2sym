% example13_emath.tex   Author: Akira Hakuta, Date: 2017/06/11
% platex.exe -kanji=utf8 -shell-escape -src -interaction=nonstopmode example13_emath.tex
% pythontex.exe example13_emath.tex
% platex.exe -kanji=utf8 -shell-escape -src -interaction=nonstopmode example13_emath.tex
% dvipdfmx.exe -p 182mm,257mm example13_emath.dvi

\documentclass[b5paper,fleqn,10pt]{jsarticle}
\setlength{\oddsidemargin}{-18mm}
\setlength{\evensidemargin}{0mm}
\setlength{\topmargin}{-18mm}
\setlength{\headheight}{6mm}
\setlength{\headsep}{2mm}
\setlength{\textwidth}{170mm}
\setlength{\textheight}{230mm}
\setlength{\hoffset}{0mm}

\usepackage{color}
\usepackage{pythontex}

\begin{pycode}
import sys
sys.path.append('../')#parent directory
from sympy import *
from tex2sym_parser import tex2sym, mylatex, mylatexstyle
F=Function('F')
var('a:z') 
var('A:C')
var('X:Z') 

ans=1 #1-> add answer, 0-> no answer
if ans==1:
       anscolor='\\color{black}'
else:
       anscolor='\\color{white}'
\end{pycode}

\usepackage{emath,emathEy}%

\usepackage{fancyhdr}%

\renewcommand{\labelenumi}{\large\textgt{\fboxsep2pt\framebox[1.7zw]{\theenumi}}} 
\renewcommand{\theenumii}{\arabic{enumii}}

\def\myvspace{\vspace{24mm}}

\newcommand{\myexpr}[2]{\pyc{texexpr=#1}
	\pyc{f=tex2sym(texexpr)} 
	\hspace{-2mm}\py{'$\displaystyle '+mylatexstyle(texexpr)+'$'}\par\myvspace
	\hfill{\py{anscolor}
		\py{'(答)~~'+'$\displaystyle '+mylatex(#2(f))+'$'}}\vspace{2mm}}
	
\newcommand{\myinequality}[1]{\pyc{texexpr=#1}
	\pyc{f=tex2sym(texexpr)} 
	\hspace{-2mm}\py{'$\displaystyle '+mylatexstyle(texexpr)+'$'}\par\myvspace
	\hfill{\py{anscolor}
		\py{'(解)~~'+'$\displaystyle '+mylatex(solve_univariate_inequality(simplify(f), x, relational=False))+'$'}}\vspace{2mm}}

\begin{document}
	\pagestyle{fancy}

	\lhead{{\Large\bf 数学演習}~~~~{\large emath~~~No.~13}}
  	\chead{}
  	\renewcommand{\headrulewidth}{0.0pt}
	\cfoot{}
  	\renewcommand{\footrulewidth}{0pt}  

	\begin{enumerate}
		
		 \item 次の式を計算せよ。
		 	\begin{edaenumerate}<3>(-2zw)[(1)]			 		
		 		\item \myexpr{r'2a^3\times 5a^6'}{simplify}
		 		\item \myexpr{'(3a^2)^4'}{simplify}
		 		\item \myexpr{'(-2x^3y^4)^5'}{simplify}
		 	\end{edaenumerate}
	 	
	 	\item 次の式を展開せよ。
		 	\begin{edaenumerate}(-2zw)[(1)]			 		
		 		\item \myexpr{'(2x+3)(4x-5)'}{expand}
		 		\item \myexpr{'\\left(\\frac{1}{~2~}\\alpha -\\frac{1}{~3~}\\beta+\\frac{1}{~4~}\\gamma \\right)^2'}{expand}	 
		 	\end{edaenumerate}
	 	
	 	\item 次の式を因数分解せよ。
		 	\begin{edaenumerate}(-2zw)[(1)]			 		
		 		\item \myexpr{'3x^2-7x-6'}{factor}
		 		\item \myexpr{'a^3+b^3+c^3-3abc'}{factor}
	 		\end{edaenumerate}
 	
 		\item 次の不等式を解け。
	 		\begin{edaenumerate}(-2zw)[(1)]			 		
	 			\item \myinequality{'x^2-3x-4 > 0'}
	 			\item \myinequality{'x^3-2x^2-5x+6 \\geqq 0'}
	 			\item \myinequality{'3x^2-2x+7 < 0'}
	 			\item \myinequality{'\\left|\,3x-4\,\\right|\\leqq 5'}	 			
	 		\end{edaenumerate}
 	
	\end{enumerate}

\end{document}


